\chapter{机器学习中的损失函数}

\section{回归问题的损失函数}

常见的回归问题的损失函数包括:
\begin{itemize}
  \item 平方损失函数 (square loss 或 l2-loss): $L(\hat{y}, y) = (\hat{y} - y)^2$
  \item 绝对值损失函数 (absolute loss 或者 l1-loss):$L(\hat{y}, y) = |\hat{y} - y|$
  \item Huber损失函数:当$|\hat{y} - y|\leq\delta$,$L(\hat{y}, y)$为$(\hat{y} - y)^2$;当$|\hat{y} - y|>\delta$,$L(\hat{y}, y)$为$|\hat{y} - y|$
\end{itemize}

平方损失函数 (square loss)最常见且具有光滑可导、凸性等优点,便于使用各种优化器。但平方损失函数的缺点在于,它对异常点 (outliers) 敏感。这里,敏感指的是由于平方损失函数的形式,异常点会最终“贡献”较大的损失值,从而导致模型在训练过程中过于受异常点的影响。我们可以以最大似然的角度解释平方损失函数:对于样本点$\bm{x}_i$,样本标签$y_i$的噪声$\epsilon^{(i)}$独立同分布,且服从于均值为0,标准差为$\delta$的高斯分布$\mathcal{N}(0,\delta)$。那么在整个训练集上的似然为:
\begin{equation}
  \begin{aligned}
    L(\bm{\theta}) &= \prod_{i=1}^{n}{p(y_i|\bm{x}_i;\bm{\theta})}
  \end{aligned}
\end{equation}

\section{分类问题的损失函数}

\subsection{二分类问题的损失函数}

最符合人类思维的二分类损失函数即为0-1 loss:
\begin{equation}
  L(\hat{y}, y) = \mathbb{I}[\hat{y} \neq y]
\end{equation}

\subsection{多分类问题的损失函数}

\section{能否直接用线性回归用在二分类任务上}

可以,效果不会特别差,但也存在一定的问题考虑margin $yf$,该value较大时,反而会贡献比较大的loss。